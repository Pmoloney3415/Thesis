\chapter{The Interaction of Light with Plasma}%
\label{chap:Theory}

This chapter shall introduce theoretical background relevent to the work conducted in this thesis.
The main focus of the work is centered on improving the laser modelling in the \textsc{Chimera} \ac{Rad-MHD} code.
Therefore, the theoretical framework for modelling both the plasma and the interaction of light with it is introduced.

Initially, the plasma state is defined and important length- and time-scales are provided.
The kinetic and fluid descriptions of plasma are introduced and the validity domain of each framework is discussed, with particular reference to typical conditions for laser-plasma interactions.
Additional physics packages of the fluid code, \textsc{Chimera}, which are utilised in later chapters, are introduced.
A model for kinetic heatflow is desirable in fluid codes which model laser-plasma interactions, therefore although \textsc{Chimera} does not have this capability, some basic theory of kinetic heatflow is highlighted.

An additional aim of the work was to include a \ac{CBET} model into the new \textsc{Chimera} laser package.
\ac{LPIs} such as \ac{CBET} are multi-wave coupling phenomena, and therefore a basic description of waves in plasma is provided.
The dispersion relation of the light waves in plasma is also derived.
Beginning with the full wave equation and then introducing successive assumptions which are broadly satiisfied in typical laser-produced \ac{ICF} plasmas, the equations of ray-tracing are derived.
Important absorption processes are then outlined, particularly \ac{Inv-Brem}, which is the dominant mechanism on the largest \ac{ICF} facilities in the world today.
Finally, the basic theory of \ac{LPIs} is provided, particularly \ac{CBET} and its relevance for direct-drive \ac{ICF}.

\newpage

%###############################################################################################################################
%###############################################################################################################################
%###############################################################################################################################
\section{Basic Plasma Physics}%
\label{sec:theory_plasma_phys}

As stated in Chap.~\ref{chap:intro}, thermonuclear fusion requires the fuel to exist at significant temperatures, which are well above ionisation energies.
Therefore, the fuel configuration in these fusion expeiments is a plasma.
Formally, a plasma is defined as a quasi-neutral, ionised gas which exhibits collective behaviour.
The charged particles within a plasma interact via the long-range Coulomb force, and thus undergo many simultaneuos interactions with the other particles.
This leads to a variety of collective phenomena such as the plasma-waves described in Sec.~\ref{sec:theory_waves_plasmas}.
Quasineutrailty of the plasma means that, when observed at a length-scale $L$, the plasma has no net charge,
\begin{equation}
    \Sigma_{\alpha}q_{\alpha}N_{\alpha} = 0,
\end{equation}
where $N_{\alpha}$ is the number of particles of species $\alpha$, with charge $q_{\alpha}$, in the cube with volume $V=L^3$.
For a `single-species' plasma\footnote{Single-species here means that there is only a single type of ion.} with average ionisation state $Z$, this implies,
\begin{equation}
    n_e = Z n_i,
\end{equation}
where $n_e$ and $n_i$ are the number densities of electrons and ions respectively.

Quasineutrailty because the particles in the plasma are free to move due to forces they experience.
Thus if a local charge imbalance occurs, the electrons, which respond faster than the ion population due to their lower mass, move to rebalance this field and restore quasineutrailty.
This electron relocation to eliminate local electric fields is known as Debye-screening.
The length scale below the electron population cannot effectively screen the charges sets the length scale of quasineutrality and is known as the Debye-length,
\begin{equation}
    \lambda_{D}^2 = \frac{\epsilon_0 k_B T_e}{n_e e^2},
\end{equation}
where $T_e$ is the electron temperature, $\epsilon_0$ is the permitivity of free space, $k_B$ is the Boltzmann constant and $e$ is the electron charge.
This is valid if there is a large number of particles in the Debye sphere, $n_e L^3\gg 1$.
The timescale of the charge relocation is of particular importance to the interaction of light with plasmas.
Light waves are an oscillating electric field, which charged particles in the plasma can respond to.
If the particles are able to respond quickly enough, they can therefore influence the propagation of the light and ultimately the plasma can become opaque.
The oscillation timescale can be derived by considering a uniform assembly of quasineutral plasma and then displacing the electron population from the ions by a small distance $\delta x$ along the $x$-axis.
The electric field which develops within the plasma is thus,
\begin{equation}
    E_x = \frac{n_e e \delta x}{\epsilon_0},
\end{equation}
leading to a restoring force $F_x = -eE_x$ on each electron.
Solving Newton's second Law demontsrates that, when thermal motion of the electron population is ignored\footnote{Inclusion of thermal motion leads to pressure, which acts as an restoring force, and yields the dispersion relation for an \ac{EPW}.}, oscillations of the electrons occur at the `plasma frequency',
\begin{equation}
    \label{eq:theory_plasma_freq}
    \omega_{p}^2 = \frac{n_e e^2}{m_e \epsilon_0}.
\end{equation}
If forcing oscillations occur at a frequency which is lower than $\omega_p$, then the electrons move rapidly enough to nullify the field.
Consider light with frequency $\omega$, which is incident normal to a plasma density gradient
Because $\omega_p\propto n_e$, the light is able to propagate until it reaches the density where $\omega = \omega_p$, which is known as the ciritcal density,
\begin{equation}
    \label{eq:theory_critical_density}
    n_{\text{cr}} = \frac{m_e \epsilon_0 \omega^2}{e^2}.
\end{equation}
After reaching the critical density, the field of the light decays exponentially as an evanescent wave, but cannot propagate.

%###############################################################################################################################
%###############################################################################################################################
%###############################################################################################################################
\section{Kinetic and Fluid Formulations of Plasmas}%
\label{sec:theory_kin_fluid_plasmas}

Idealised computational modelling of a plasma state would solve the long-range electro-magnetic interaction between every pair of particles at all times.
However, this very rapidly becomes intractable due to the large number $N$ of particles and the $\mathcal{O}(N^2)$ scaling of interactions to solve.
Reduced frameworks must thus be devised with which to analyse and predict the behaviour of the plasma state.
Plasmas are divided into two broad classifications.
When in local thermal equilibrium, the plasma is often described as `thermal' and the fluid formulation is an adequate description.
When this is not true however, for instance a particular subset of particles are heated at a rate, which is much faster than thermalising collision timescales, this subset of the system is described as `non-thermal' ot kinetic.
In this case the fluid formulation is an inadequate description and higher fidelity tools must be used to describe the evloution of the system.

When a kinetic description of a plasma is required, the distribution function, $f_{\alpha}(\vec{x},\vec{v},t)$ is used to describe the state of particle species $\alpha$.
It provides a statistical description of the number density of particles, which inhabit a phase-space, $\vec{x}$-$\vec{v}$, at time $t$.
If the system evolves on a time-scale much lower than the collision time, then these collisions between particles act to relax the distribution function toward a Maxwellian,
\begin{equation}
    f_{\alpha,\text{Max.}}(v) = n_{\alpha} {\left( \frac{1}{2\pi v_{\text{th}}^2} \right)}^{1.5} e^{-v^2/(2 v_{\text{th}}^2)},
\end{equation}
where $v_{\text{th}}=\sqrt{k_B T_{\alpha}/m_{\alpha}}$ is the thermal speed of the species with temperature and mass $T_{\alpha}$ and $m_{\alpha}$, respectively.
The fraction outside the exponential is set such the integral over velocity space yields the number density of the species.
Typically, for bulk of the target configuration throughout a laser-produced \ac{ICF} implosion, the assumption of a Maxwellian distribution is close to accurate although there are notable exceptions.
For instance, DT fusion products have energies much higher than thermal energies and are also monoenergetic.
\ac{LPIs} also generate energetic electron populations which are able to range through the implosion due to their low collisionality.
Laser-heated plasmas also typically exhibit steep density and temperature gradients near the ablation surface, such that collisions, and therefore collision processes such as transport of thermal energy, do not act locally.

%################################################################################
%################################################################################
\subsection{The Vlasov Equation}%
\label{sec:theory_vlasov}

The evolution of the distribution function for each species indiviudally, is described by the Vlasov equation,
\begin{equation}
    \label{eq:theory_vlasov}
    \frac{\partial f_\alpha}{\partial t} + \vec{v}.\nabla_x f_\alpha + \frac{q_\alpha}{m_\alpha} (\vec{E} + \vec{v}\times\vec{B}) . \nabla_v f_\alpha = \left ( \frac{\partial f_\alpha}{\partial t} \right )_\text{collisions},
\end{equation}
where $q_\alpha$ is the charge of the species, $\vec{E}$ and $\vec{B}$ are the macroscopic electric and magnetic fields\footnote{Macroscopic here means on a scale larger than Debye shielding.}, respectively, and $\nabla_x$ and $\nabla_v$ are gradients with respect to postiion and velocity coordinates respectively.
The equation describes the conservation of phase-space particle density.
The collision operator on the right hand side of Eq.~\ref{eq:theory_vlasov} describes the action of microscopic fields, which arise due to the random motion of the charged plasma particles.
Evolution of the macroscopic fields is governed by Maxwell's equations,
\begin{align}
    \label{eq:theory_maxwell_eqs}
    \nabla.\vec{E} &= \fra{\rho_{\text{charge}}}{\epsilon_0},\\
    \nabla.\vec{B} &= 0,\\
    \nabla \times \vec{E} &= -\frac{\partial \vec{B}}{\partial t},\\
    \nabla \times \vec{E} &= \frac{1}{\mu_0 \epsilon_0}\frac{\partial \vec{E}}{\partial t} + \mu_0 \vec{j},\\
\end{align}
where $\mu_0$ is the permeability of free space, $\rho_{\text{charge}}$ is the charge density and $\vec{j}$ is the current density.

x and v of all particles described by f.
bulk collision operator and lorentz force give vlasov equation.
Solved by VFP.

%################################################################################
%################################################################################
\subsection{The Fluid Equations}%
\label{sec:theory_fluid}

Take moments of vlasov, assuming equilibrium, to get fluid equations.
Give assumptions.
Describe closure problem.
Say how solved in Chimera.
Knudsen number describes how kinetic it is.

%################################################################################
%################################################################################
\subsection{Radiation Transport}%
\label{sec:theory_radtransp}

Additional mechanism which described emission and absorption of radiation by plasma.
Important for high temperatures and densities of ICF.
Say how solved in Chimera.

%################################################################################
%################################################################################
\subsection{Magnetohydrodynamics}%
\label{sec:theory_MHD}

Say electric fields on large scales don't exist in plasma because they are dielectric.
If magnetic field are important, then extend the fluid equations to include the effect of B on the above equations.

%##########################################################
\subsubsection{Ideal MHD}%
\label{sec:theory_idealMHD}

Get lorentz force on plasma.
Solve for B using Maxwell.
Solve for j using Ohms law.
Magnetic tension and pressure.
Give plasma beta.

%##########################################################
\subsubsection{Magnetised Heatflow}%
\label{sec:theory_magheatflow}

Give Hall parameters.
Say they describe transport processes like heatflow.

%##########################################################
\subsubsection{Resistive MHD}%
\label{sec:theory_resisMHD}

Plasma often not perfectly conducting so B field moves without flow.
Describe Rm.
Say when important.

%##########################################################
\subsubsection{The Nernst Effect}%
\label{sec:theory_nernst}

Hmmm, describe and say when important.

%##########################################################
\subsubsection{The Biermann Battery Effect}%
\label{sec:theory_biermann}

Hmmm, describe and say when important.

%################################################################################
%################################################################################
\subsection{Kinetic Heatflow}%
\label{sec:theory_kineticheatflow}

Say that in laser heated plasmas, often have Knudsen significant due to high temps and low densities.
Therefore, accurate modes include kinetic model for heatflux.
Either VFP or SNB etc, say roughly what they do.

%###############################################################################################################################
%###############################################################################################################################
%###############################################################################################################################
\section{Waves in Plasma}%
\label{sec:theory_waves_plasmas}

3 waves can exist in plasma without B.
First look at plasma waves, i.e. not a light wave, both of which are longitudinal.

%################################################################################
%################################################################################
\subsection{Plasmas as a Dielectric Medium}%
\label{sec:theory_dielectric}

Talk about how plasmas are dielectric, therefore described by susceptibilties.
Can get dispersion relations by susceptibilities and outline process.
Multi-species effects.

%################################################################################
%################################################################################
\subsection{Plasma Waves}%
\label{sec:theory_longwaves}

Get disp rels of IAW and EPW and say what they both are physically.

%################################################################################
%################################################################################
\subsection{Light Waves}%
\label{sec:theory_transwaves}

These are transverse waves and therefore don't create space-charge separation, unlike longitudinal waves.
Get disp rel and show wpe comes out.
Give physical interpretation.

%###############################################################################################################################
%###############################################################################################################################
%###############################################################################################################################
\section{Propagation of Light in a Plasma}%
\label{sec:theory_propagation}

Want to described light propagating through plasma in limit of typical ICF configurations.
Weakly focussing, moderate intensities etc.


%################################################################################
%################################################################################
\subsection{Paraxial Approximation}%
\label{sec:theory_paraxial}

Weakly focussing limit.
Give equations, interpretations and validity.

%################################################################################
%################################################################################
\subsection{WKB Approximation}%
\label{sec:theory_WKB}

Uniform medium limit.
Give equations, interpretations and validity.
Give Airy example - ie not valid nearby turning point.

%################################################################################
%################################################################################
\subsection{Ray Tracing}%
\label{sec:theory_rays}

Give equations, interpretations and validity.
State can be used for any kind of wave where valid.
Say how and why used for direct drive and typical frozen plasma assumption.

Talking about validity region, give extra bits can solve like ray amplitude.


%###############################################################################################################################
%###############################################################################################################################
%###############################################################################################################################
\section{Absorption of Light in a Plasma}%
\label{sec:theory_absorption}

ICF we want to give laser energy to plasma to drive implosion, therefore need to talk about absorption.

%################################################################################
%################################################################################
\subsection{Inv Brem}%
\label{sec:theory_in_brem}

Introduce all bits to get NRL formaularly equation.
Talk about Langdon as well.

%################################################################################
%################################################################################
\subsection{Resonance Absorption}%
\label{sec:theory_res_abs}

Introduce

%################################################################################
%################################################################################
\subsection{ICF Relevant Absorption comparison}%
\label{sec:theory_absorption_comparison}

Say inv brem goes up relatively at larger scales and shorter wavelengths.
Prefered to resonance absorption because bulk population gets energy.
Therefore use frequency tripled light.


%###############################################################################################################################
%###############################################################################################################################
%###############################################################################################################################
\section{Laser Plasma instabilities}%
\label{sec:theory_LPIs}

%################################################################################
%################################################################################
\subsection{Ponderamotive Force}%
\label{sec:theory_ponderamotive}

Introduce and say why it happens roughly.

%################################################################################
%################################################################################
\subsection{Three-Wave Coupling}%
\label{sec:theory_threewave}

Give the general picture, i.e. ponderamotive, perturbation, driven plasma wave.
Give momentum and energy conservation.
List all types seeded by an EMW.

%################################################################################
%################################################################################
\subsection{Cross-Beam Energy Transfer}%
\label{sec:theory_CBET}

Derive something to an appropriate level of detail.
Talk about how it is in frame of plasma, flow velocities change this, mach 1 surface etc.
Give general picture, i.e. sidescatter and backscatter and what it does in ICF.


%##########################################################
\subsubsection{Linear Gain Theory}%
\label{sec:theory_lineargaincbet}

Say that we use this for raytracing.
Assume uniform plasma and can solve plasma response either by linearising fluid or kinetic equations.

%##########################################################
\subsubsection{Effect of Polarisation}%
\label{sec:theory_cbet_polarisation}

Say that LPIs are affected by polarisation via ponderatmotive beat.
Only parallel polarisations interact.
Important on OMEGA due to polarisation smoothing, leads to mode-1.

%##########################################################
\subsubsection{Langdon Effect on CBET}%
\label{sec:theory_cbet_langdon}

Say that Langdon affects cbet.
Reduced model to alter linear gain.
Could potentially explain why indirect ICF models require a clamp.

%################################################################################
%################################################################################
\subsection{Mitigation of Laser Plasma Instabilities}%
\label{sec:theory_lpi_mitigation}

Say its coherence spatially, temporally and spectrally, so break this to stop LPIs.
Mention stud pulses and say zooming for CBET.

Mainstream approach is bandwidth.
Talk about studies showing that bandwidth should mitigate CBET.
Talk about experimental progress eg FLUX laser at LLE.


%###############################################################################################################################
%###############################################################################################################################
%###############################################################################################################################
\section{Summary}%
\label{sec:theory_summary}

Summarise that introduced descriptions of plasmas, particularly fluid framework solved by CHIMERA.
Talked about waves in plasmas and their physical interpareatations.
Talked about how light propagates, assumptions etc, used for raytracing in next section.
LPIs, particularly CBET, modelling it is focus of next chapter.
