\chapter{Theory}%
\label{chap:Theory}


%###############################################################################################################################
%###############################################################################################################################
%###############################################################################################################################
\section{Basic Plasma Physics}%
\label{sec:theory_plasma_phys}

What is a plasma.
Quasineutrality.
Debeye screening.
Plasma Frequency derivation.
Turn this into critical density and say this is importatn for lasers.

%###############################################################################################################################
%###############################################################################################################################
%###############################################################################################################################
\section{Kinetic and Fluid Formulations of Plasmas}%
\label{sec:theory_kin_fluid_plasmas}

A bit of blather about what these descriptions are trying to do generally.
Ideally would solve every interaction between every particle but this is intractible.

%################################################################################
%################################################################################
\subsection{The Vlasov Equation}%
\label{sec:theory_vlasov}

x and v of all particles described by f.
bulk collision operator and lorentz force give vlasov equation.
Solved by VFP.

%################################################################################
%################################################################################
\subsection{The Fluid Equations}%
\label{sec:theory_fluid}

Take moments of vlasov, assuming equilibrium, to get fluid equations.
Give assumptions.
Describe closure problem.
Say how solved in Chimera.
Knudsen number describes how kinetic it is.

%################################################################################
%################################################################################
\subsection{Radiation Transport}%
\label{sec:theory_kineticheatflow}

Additional mechanism which described emission and absorption of radiation by plasma.
Important for high temperatures and densities of ICF.
Say how solved in Chimera.

%################################################################################
%################################################################################
\subsection{Magnetohydrodynamics}%
\label{sec:theory_MHD}

Say electric fields on large scales don't exist in plasma because they are dielectric.
If magnetic field are important, then extend the fluid equations to include the effect of B on the above equations.

%##########################################################
\subsubsection{Ideal MHD}%
\label{sec:theory_idealMHD}

Get lorentz force on plasma.
Solve for B using Maxwell.
Solve for j using Ohms law.
Magnetic tension and pressure.
Give plasma beta.

%##########################################################
\subsubsection{Magnetised Heatflow}%
\label{sec:theory_magheatflow}

Give Hall parameters.
Say they describe transport processes like heatflow.

%##########################################################
\subsubsection{Resistive MHD}%
\label{sec:theory_resisMHD}

Plasma often not perfectly conducting so B field moves without flow.
Describe Rm.
Say when important.

%##########################################################
\subsubsection{The Nernst Effect}%
\label{sec:theory_resisMHD}

Hmmm, describe and say when important.

%##########################################################
\subsubsection{The Biermann Battery Effect}%
\label{sec:theory_biermann}

Hmmm, describe and say when important.

%################################################################################
%################################################################################
\subsection{Kinetic Heatflow}%
\label{sec:theory_kineticheatflow}

Say that in laser heated plasmas, often have Knudsen significant due to high temps and low densities.
Therefore, accurate modes include kinetic model for heatflux.
Either VFP or SNB etc, say roughly what they do.

%###############################################################################################################################
%###############################################################################################################################
%###############################################################################################################################
\section{Waves in Plasma}%
\label{sec:theory_waves_plasmas}

3 waves can exist in plasma without B.
First look at plasma waves, i.e. not a light wave, both of which are longitudinal.

%################################################################################
%################################################################################
\subsection{Plasmas as a Dielectric Medium}%
\label{sec:theory_dielectric}

Talk about how plasmas are dielectric, therefore described by susceptibilties.
Can get dispersion relations by susceptibilities and outline process.
Multi-species effects.

%################################################################################
%################################################################################
\subsection{Plasma Waves}%
\label{sec:theory_longwaves}

Get disp rels of IAW and EPW and say what they both are physically.

%################################################################################
%################################################################################
\subsection{Light Waves}%
\label{sec:theory_transwaves}

These are transverse waves and therefore don't create space-charge separation, unlike longitudinal waves.
Get disp rel and show wpe comes out.
Give physical interpretation.

%###############################################################################################################################
%###############################################################################################################################
%###############################################################################################################################
\section{Propagation of Light in a Plasma}%
\label{sec:theory_propagation}

Want to described light propagating through plasma in limit of typical ICF configurations.
Weakly focussing, moderate intensities etc.


%################################################################################
%################################################################################
\subsection{Paraxial Approximation}%
\label{sec:theory_paraxial}

Weakly focussing limit.
Give equations, interpretations and validity.

%################################################################################
%################################################################################
\subsection{WKB Approximation}%
\label{sec:theory_WKB}

Uniform medium limit.
Give equations, interpretations and validity.
Give Airy example - ie not valid nearby turning point.

%################################################################################
%################################################################################
\subsection{Ray Tracing}%
\label{sec:theory_rays}

Give equations, interpretations and validity.
State can be used for any kind of wave where valid.
Say how and why used for direct drive and typical frozen plasma assumption.

Talking about validity region, give extra bits can solve like ray amplitude.


%###############################################################################################################################
%###############################################################################################################################
%###############################################################################################################################
\section{Absorption of Light in a Plasma}%
\label{sec:theory_absorption}

ICF we want to give laser energy to plasma to drive implosion, therefore need to talk about absorption.

%################################################################################
%################################################################################
\subsection{Inv Brem}%
\label{sec:theory_in_brem}

Introduce all bits to get NRL formaularly equation.
Talk about Langdon as well.

%################################################################################
%################################################################################
\subsection{Resonance Absorption}%
\label{sec:theory_res_abs}

Introduce

%################################################################################
%################################################################################
\subsection{ICF Relevant Absorption comparison}%
\label{sec:theory_absorption_comparison}

Say inv brem goes up relatively at larger scales and shorter wavelengths.
Prefered to resonance absorption because bulk population gets energy.
Therefore use frequency tripled light.


%###############################################################################################################################
%###############################################################################################################################
%###############################################################################################################################
\section{Laser Plasma instabilities}%
\label{sec:theory_LPIs}

%################################################################################
%################################################################################
\subsection{Ponderamotive Force}%
\label{sec:theory_ponderamotive}

Introduce and say why it happens roughly.

%################################################################################
%################################################################################
\subsection{Three-Wave Coupling}%
\label{sec:theory_threewave}

Give the general picture, i.e. ponderamotive, perturbation, driven plasma wave.
Give momentum and energy conservation.
List all types seeded by an EMW.

%################################################################################
%################################################################################
\subsection{Cross-Beam Energy Transfer}%
\label{sec:theory_CBET}

Derive something to an appropriate level of detail.
Talk about how it is in frame of plasma, flow velocities change this, mach 1 surface etc.
Give general picture, i.e. sidescatter and backscatter and what it does in ICF.


%##########################################################
\subsubsection{Linear Gain Theory}%
\label{sec:theory_lineargaincbet}

Say that we use this for raytracing.
Assume uniform plasma and can solve plasma response either by linearising fluid or kinetic equations.

%##########################################################
\subsubsection{Effect of Polarisation}%
\label{sec:theory_cbet_polarisation}

Say that LPIs are affected by polarisation via ponderatmotive beat.
Only parallel polarisations interact.
Important on OMEGA due to polarisation smoothing, leads to mode-1.

%##########################################################
\subsubsection{Langdon Effect on CBET}%
\label{sec:theory_cbet_langdon}

Say that Langdon affects cbet.
Reduced model to alter linear gain.
Could potentially explain why indirect ICF models require a clamp.

%################################################################################
%################################################################################
\subsection{Mitigation of Laser Plasma Instabilities}%
\label{sec:theory_lpi_mitigation}

Say its coherence spatially, temporally and spectrally, so break this to stop LPIs.
Mention stud pulses and say zooming for CBET.

Mainstream approach is bandwidth.
Talk about studies showing that bandwidth should mitigate CBET.
Talk about experimental progress eg FLUX laser at LLE.


%###############################################################################################################################
%###############################################################################################################################
%###############################################################################################################################
\section{Summary}%
\label{sec:theory_summary}

Summarise that introduced descriptions of plasmas, particularly fluid framework solved by CHIMERA.
Talked about waves in plasmas and their physical interpareatations.
Talked about how light propagates, assumptions etc, used for raytracing in next section.
LPIs, particularly CBET, modelling it is focus of next chapter.
