\begin{abstract}

Accurate modelling of laser propagation and energy deposition in direct-drive \ac{ICF} is crucial to performing predictive simulations.
Laser energy drives all subsequent dynamics, therefore unless a valid model is employed to obtain the deposition profile, the capability of codes to both design future implosions and analyse past experiments is significantly reduced.
\ac{CBET} is an interaction of laser light and the plasma state of matter, which can halve the deposited power in direct-drive implosions and amplify deposition asymmetries by an order of magnitude.
Computational models which are used to simulate direct-drive implosions and do not include this effect, therefore cannot be truly predictive.
This thesis presents the development, validation and use of \textsc{Solas}: a 3-D, ray-based \ac{CBET} model, which has been integrated into the \textsc{Chimera} code.

The development of \textsc{Solas} and its integration into the \textsc{Chimera} code are described.
A novel computational grid for the ray-trace, which allows for adaptive radial resolution and straightforward coupling to an Eulerian hydrodynamics code, is presented.
Test problems were conducted which validated the ray-trace, energy deposition, electric field reconstruction and \ac{CBET} solver.
\textsc{Chimera}-\textsc{Solas} simulations of direct-drive targets on the \textsc{Omega} laser facility are presented, which are in excellent agreement with existing codes.

A study is presented, which investigated the role of \ac{CBET} with respect to the beam radius initial condition for direct-drive implosions.
A 2-D cylindrical simulation platform was developed, which allowed for rapid computation of an ensemble of both \ac{CBET} and no-\ac{CBET} calculations.
The results of this study demonstrate that in the absence of \ac{CBET}, increasing the beam radius improves the stagnation state symmetry of implosions.
However, a larger beam radius leads to more \ac{CBET} and a subsequent degradation of symmetry.
These results could help to explain observed trends in statistical modelling of \textsc{Omega} implosions.

A final study is presented, which aimed to understand how the role of \ac{CBET} changes when a target is magnetised prior to the laser drive.
Pre-magnetisation is predicted to reduce hotspot thermal conduction losses, and could thus potentially enhance the maximum performance of implosions at a given laser energy.
However, magnetised direct-drive experiments have demonstrated that anisotropic thermal conduction in the coronal plasma can lead to additional asymmetry of the implosion.
\ac{CBET} is known to be highly sensitive to long wavelength, coronal asymmetries.
Therefore, simulations of magnetised direct-drive targets were conducted, to understand if magnetisation of the corona affected \ac{CBET} scattering.
While \ac{CBET} was observed to be dynamically significant in these implosions, significantly reducing the coupled energy, its effect appeared to be mostly insensitive to the level of pre-magnetisation.
Particularly, the stagnation state asymmetry was very similar for simulations when \ac{CBET} redistribution of deposition was included and discounted.
Modifications to the implosion design are suggested, which may lead to a more experimentally observable impact of magnetisation on \ac{CBET}.

It is hoped that the next generation of direct-drive laser facilities will incorporate bandwidth to fully mitigate \ac{CBET}.
Therefore, understanding how \ac{CBET} affects current implosions is crucial to be able to confidently extrapolate performance onto these future facilities.
3-D \ac{CBET} models, which are able to model both the energy coupling losses and enhanced deposition asymmetry, are therefore a crucial tool for experimental analysis and design.

\end{abstract}