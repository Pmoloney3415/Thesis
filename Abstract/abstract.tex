\begin{abstract}

Accurate modelling of laser propagation and energy deposition in direct-drive \ac{ICF} is crucial to performing predictive simulations.
\ac{CBET} is an interaction of laser light and the plasma state of matter, which can halve the deposited power in direct-drive implosions and amplify deposition asymmetries by an order of magnitude.
Computational models which do not account for \ac{CBET}, can therefore not be truly predictive.
This thesis presents the development, validation and use of \textsc{Solas}: a 3-D, ray-based \ac{CBET} model, which has been integrated into the \textsc{Chimera} code.

The development of \textsc{Solas} and its integration into the \textsc{Chimera} code are described.
Test problems were conducted which validated the ray-trace, energy deposition, electric field reconstruction and \ac{CBET} solver.
\textsc{Chimera}-\textsc{Solas} simulations of direct-drive targets on the \textsc{Omega} laser facility are presented, which are in excellent agreement with existing codes.

A study is presented, which investigated the role of \ac{CBET} with respect to the beam radius initial condition for direct-drive implosions.
The results demonstrate that in the absence of \ac{CBET}, increasing the beam radius improves the stagnation state symmetry of implosions.
However, a larger beam radius leads to more \ac{CBET} and a subsequent symmetry degradation.
These results could help to explain observed trends in statistical modelling of \textsc{Omega} implosions.

A final study is presented, which aimed to understand how the role of \ac{CBET} changes when a target is magnetised prior to the laser drive.
Magnetised direct-drive experiments have demonstrated that anisotropic thermal conduction in the coronal plasma can lead to asymmetry of the implosion.
Simulations of magnetised direct-drive targets were conducted, to understand if coronal magnetisation affected \ac{CBET} scattering.
Although \ac{CBET} was observed to be dynamically significant to the implosions, the interplay of \ac{CBET} with the magnetic field was found to be minimal.

\end{abstract}